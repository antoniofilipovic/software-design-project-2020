\chapter{Zaključak i budući rad}
		\normalfont Kada smo tek krenuli izrađivati projekt, najveći blagoslov, ali i teret na neki način predstavljalo je što za izradu imamo otvorene sve mogućnosti, ali trebalo je na neki način ukomponirati da izrada ne bude previše komplicirana, a da tehnologije koje koristimo nam omoguće laganu nadogradnju za kasnije. Uz to trebalo je i riješiti sve zadatke zadane. Odabrali smo Spring Boot s obzirom da doista uvelike olakšava izradu web aplikacija jer toliko stvari rješava u pozadini da je bilo malo i zastrašujuće, s obzirom da je nas nekolicina navikla bila pisati sve samostalno. Sljedeći izazov bio je frontend, točnije koju tehnologiju odabrati. Sad kad je aplikacija gotova mišljenja sam da ono što smo mi odabrali, a to je JSP, tehnologija je kod koje je užasno teško za ukomponirat više stvari na stranici.  Bilo bi lakše da smo odabrali React iako ga nitko ne zna u grupi. Mnogo bi naučili, ali ostavit ćemo ga za sljedeći put.\\
		Također trebalo je osmisliti kako ćemo u bazi čuvati podatke te kako ćemo to dalje slati prema JSP stranicama koje te podatke koriste. Organiziranje baze nije bio lagan posao, možda smo nešto mogli drugačije, ali kako trenutno stvari stoje, baza se lagano može nadogradit ako zatrebaju nove funkcionalnosti, tako da smatram da je taj dio uspješno riješen.\\
		Znanja koja su potrebna za bržu izradu projekta je definitivno poznavanje objektno orjenitrane paradigme te općenita znanja za baze podataka, te definitivno poznavanje HTML-a te Javascripta. Jako bi nam bilo otežano da smo sve stvari pisali kad god nam je u kodu bilo potrebno.  No i taj dio smo dobro riješili s obzirom da smo napravili dobar reusability u kodu.\\
		Također kod se može isto lagano nadograditi za nove funkcionalnosti, tipa novu vrstu korisnika, ali to bi vjerojatno zakompliciralo stvari na frontendu pa zbog toga React bi bio bolja opcija.\\
		Uz to svi smo naučili koristiti se Git-om. U početku je to predstavljalo problem, ali kad smo savladali par puta što se može desiti, lako smo kasnije rješavali probleme.\\
		Još jednom, sve funkcionalnosti su implemenitrane.
		
		
		\eject 